\documentclass[11pt,twoside,printwatermark=false]{pinp}

%% Some pieces required from the pandoc template
\providecommand{\tightlist}{%
  \setlength{\itemsep}{0pt}\setlength{\parskip}{0pt}}

% Use the lineno option to display guide line numbers if required.
% Note that the use of elements such as single-column equations
% may affect the guide line number alignment.

\usepackage[T1]{fontenc}
\usepackage[utf8]{inputenc}

\definecolor{pinpblue}{HTML}{185FAF}  % imagecolorpicker on blue for new R logo
\definecolor{pnasbluetext}{RGB}{101,0,0} %



\title{EEE933 - Design and Analysis of Experiments}

\author[]{Case Study 01}


\setcounter{secnumdepth}{0}

% Please give the surname of the lead author for the running footer
\leadauthor{}

% Keywords are not mandatory, but authors are strongly encouraged to provide them. If provided, please include two to five keywords, separated by the pipe symbol, e.g:
 

\begin{abstract}
Experiment: performance of a new software version.
\end{abstract}

\dates{This version was compiled on \today}


\begin{document}

% Optional adjustment to line up main text (after abstract) of first page with line numbers, when using both lineno and twocolumn options.
% You should only change this length when you've finalised the article contents.
\verticaladjustment{-2pt}

\maketitle
\thispagestyle{firststyle}
\ifthenelse{\boolean{shortarticle}}{\ifthenelse{\boolean{singlecolumn}}{\abscontentformatted}{\abscontent}}{}

% If your first paragraph (i.e. with the \dropcap) contains a list environment (quote, quotation, theorem, definition, enumerate, itemize...), the line after the list may have some extra indentation. If this is the case, add \parshape=0 to the end of the list environment.


\section{The experiment}\label{the-experiment}

The \textbf{current} version of a given system is known, based on
extensive past experience, to have a distribution of execution costs
with populational mean \(\mu = 50\) and populational variance
\(\sigma^2 = 100\) (the specific units are not important in this
particular experiment). Notice that these are considered populational
parameters, not sample estimates.

A \textbf{new} version of this software is developed, and we wish to
investigate whether it results in \emph{performance improvements} (e.g.,
smaller mean execution cost and/or smaller variance) in relation to the
current standard. To investigate this particular question, an
experimental analysis will be performed.

To \textbf{simulate} the data collection procedures for the \textbf{new}
software version, the following routines will be used. First, to set up
the data collection procedure, use:

\begin{Shaded}
\begin{Highlighting}[]
\CommentTok{# Install required package and set up simulation}
\KeywordTok{install.packages}\NormalTok{(}\StringTok{"ExpDE"}\NormalTok{) }\CommentTok{# <-- you only need to install it once}
\end{Highlighting}
\end{Shaded}

\begin{Shaded}
\begin{Highlighting}[]
\CommentTok{# Set up the data-generating procedure}
\KeywordTok{library}\NormalTok{(ExpDE)}
\NormalTok{mre <-}\StringTok{ }\KeywordTok{list}\NormalTok{(}\DataTypeTok{name =} \StringTok{"recombination_bin"}\NormalTok{, }\DataTypeTok{cr =} \FloatTok{0.9}\NormalTok{) }
\NormalTok{mmu <-}\StringTok{ }\KeywordTok{list}\NormalTok{(}\DataTypeTok{name =} \StringTok{"mutation_rand"}\NormalTok{, }\DataTypeTok{f =} \DecValTok{2}\NormalTok{)}
\NormalTok{mpo <-}\StringTok{ }\DecValTok{100}
\NormalTok{mse <-}\StringTok{ }\KeywordTok{list}\NormalTok{(}\DataTypeTok{name  =} \StringTok{"selection_standard"}\NormalTok{)}
\NormalTok{mst <-}\StringTok{ }\KeywordTok{list}\NormalTok{(}\DataTypeTok{names =} \StringTok{"stop_maxeval"}\NormalTok{, }\DataTypeTok{maxevals =} \DecValTok{10000}\NormalTok{)}
\NormalTok{mpr <-}\StringTok{ }\KeywordTok{list}\NormalTok{(}\DataTypeTok{name  =} \StringTok{"sphere"}\NormalTok{, }\DataTypeTok{xmin  =} \OperatorTok{-}\KeywordTok{seq}\NormalTok{(}\DecValTok{1}\NormalTok{, }\DecValTok{20}\NormalTok{), }\DataTypeTok{xmax  =} \DecValTok{20} \OperatorTok{+}\StringTok{ }\DecValTok{5} \OperatorTok{*}\StringTok{ }\KeywordTok{seq}\NormalTok{(}\DecValTok{5}\NormalTok{, }\DecValTok{24}\NormalTok{))}
\end{Highlighting}
\end{Shaded}

A \textbf{single} observation of running cost for the \textbf{new}
version can be obtained by running:

\begin{Shaded}
\begin{Highlighting}[]
\CommentTok{# Generate a single observation}
\KeywordTok{ExpDE}\NormalTok{(mpo, mmu, mre, mse, mst, mpr,}
      \DataTypeTok{showpars =} \KeywordTok{list}\NormalTok{(}\DataTypeTok{show.iters =} \StringTok{"none"}\NormalTok{))}\OperatorTok{$}\NormalTok{Fbest}
\end{Highlighting}
\end{Shaded}

\begin{ShadedResult}
\begin{verbatim}
#  [1] 51.45831
\end{verbatim}
\end{ShadedResult}

\begin{center}\rule{0.5\linewidth}{\linethickness}\end{center}

\newpage

\section{Activities}\label{activities}

\subsection{For the test on the mean
cost}\label{for-the-test-on-the-mean-cost}

For this test, assume a desired significance level \(\alpha = 0.01\), a
minimally relevant effect size of \(\delta^* = 4\), and a desired power
of \(\pi = 1 - \beta = 0.8\). Each team must perform the following
activities:

\begin{itemize}
\tightlist
\item
  Define the statistical hypotheses to be tested (null/alternative).
\item
  Define the sample size to be used in this experiment, and collect the
  sample.
\item
  Test the hypotheses and decide for rejecting (or not) the null
  hypothesis.
\item
  Calculate the confidence interval on the mean.
\item
  Validate and discuss the assumptions of the test.
\item
  Discuss the power of the test (if needed), and the adequacy of the
  sample size used for this particular test.
\end{itemize}

\subsection{For the test on the variance of the
cost}\label{for-the-test-on-the-variance-of-the-cost}

For this test, assume a desired significance level \(\alpha = 0.05\).
Assume also that reductions in standard deviation is a secondary
benefit, to be investigated using the same observations collected for
the first test. Each team must perform the following activities:

\begin{itemize}
\tightlist
\item
  Define the statistical hypotheses to be tested (null/alternative).
\item
  Test the hypotheses and decide for rejecting (or not) the null
  hypothesis.
\item
  Calculate the confidence interval on the variance.
\item
  Validate and discuss the assumptions of the test.
\end{itemize}

After performing the activities related to each test individually, the
team must:

\begin{itemize}
\tightlist
\item
  Draw conclusions and provide recommendations regarding the adoption
  (or not) of the new software version.
\item
  Discuss possible ways to improve this experiment.
\end{itemize}

\subsection{{[}Bonus question{]}}\label{bonus-question}

Can you also provide a 90\% tolerance interval (using a confidence level
\(\alpha = 0.05\)) for the population of running costs? Notice that the
data may not be Normal.

\begin{center}\rule{0.5\linewidth}{\linethickness}\end{center}

\newpage

\section{Report}\label{report}

Each team must prepare a short report detailing the experiment and the
analysis performed. The report will be evaluated according to the
following criteria:

\begin{itemize}
\tightlist
\item
  Use of the predefined format (see below);
\item
  Reproducibility of the analyses;
\item
  Technical quality;
\item
  Logical structure;
\item
  Correct use of language (grammar, orthography, \emph{etc.});
\end{itemize}

The report must \textbf{necessarily} be prepared using
\href{http://rmarkdown.rstudio.com}{R Markdown}, and must contain the
full code needed to reproduce the analysis performed by the team,
embedded in the form of \emph{code blocks}. Each team must deliver the
following files:

\begin{itemize}
\tightlist
\item
  The report file, compiled in \textbf{.pdf} .
\item
  The original (\textbf{.Rmd} source file) of the report.
\item
  The data file (\textbf{.csv}) generated in this experiment.
\end{itemize}

The \textbf{.Rmd} file must be able to be recompiled, if needed (tip:
save your \textbf{.Rmd} file using UTF-8 encoding, to prevent
compilation problems in other operational systems.

Report templates are available on \url{https://git.io/vHk0F}, and an
example of report structure can be consulted on
\url{https://git.io/vHk0j}. This document can also be used as a
template.

\textbf{Important}: Please include in the report the roles of each team
member (Coordinator, Recorder, Checker and, for 4-member teams, Monitor)

\textbf{Important}: Reports can be prepared in either Portuguese or
English.

\begin{center}\rule{0.5\linewidth}{\linethickness}\end{center}

\section{Deadline}\label{deadline}

The report files (pdf + rmd + csv) must be compressed into a single file
(use \textbf{.ZIP}) and uploaded to the activity \textbf{Case Study 01}
on Moodle, until \textbf{Monday, September 24 2018, 11:59p.m.}. After
that deadline the system will be closed.

\textbf{Important}: Only ONE submission is required for each team.

\textbf{Important}: Reports will NOT be received by e-mail or in printed
form.

%\showmatmethods





\end{document}

