\documentclass[11pt,twocolumn,printwatermark=false]{pinp}

%% Some pieces required from the pandoc template
\providecommand{\tightlist}{%
  \setlength{\itemsep}{0pt}\setlength{\parskip}{0pt}}

% Use the lineno option to display guide line numbers if required.
% Note that the use of elements such as single-column equations
% may affect the guide line number alignment.

\usepackage[T1]{fontenc}
\usepackage[utf8]{inputenc}

\definecolor{pinpblue}{HTML}{185FAF}  % imagecolorpicker on blue for new R logo
\definecolor{pnasbluetext}{RGB}{101,0,0} %



\title{EEE933 - Estudo de Caso 02}

\author[a]{Felipe Campelo}

  \affil[a]{Programa de Pós-Graduação em Engenharia Elétrica, UFMG.}

\setcounter{secnumdepth}{0}

% Please give the surname of the lead author for the running footer
\leadauthor{}

% Keywords are not mandatory, but authors are strongly encouraged to provide them. If provided, please include two to five keywords, separated by the pipe symbol, e.g:
 

\begin{abstract}
Comparação do IMC médio de alunos do PPGEE-UFMG ao longo de dois
semestres
\end{abstract}

\dates{This version was compiled on \today}

\pinpfootercontents{Document created using R Markdown + PINP package}

\begin{document}

% Optional adjustment to line up main text (after abstract) of first page with line numbers, when using both lineno and twocolumn options.
% You should only change this length when you've finalised the article contents.
\verticaladjustment{-2pt}

\maketitle
\thispagestyle{firststyle}
\ifthenelse{\boolean{shortarticle}}{\ifthenelse{\boolean{singlecolumn}}{\abscontentformatted}{\abscontent}}{}

% If your first paragraph (i.e. with the \dropcap) contains a list environment (quote, quotation, theorem, definition, enumerate, itemize...), the line after the list may have some extra indentation. If this is the case, add \parshape=0 to the end of the list environment.


\subsection{O Experimento}\label{o-experimento}

Neste estudo deseja-se comparar o IMC
{[}\href{http://apps.who.int/bmi/index.jsp?introPage=intro_3.html}{1}{]}
médio de duas populações de estudantes, a saber: alunos de pós-graduação
em Engenharia Elétrica na UFMG nos semestres 2016-2 e 2017-2. Neste caso
estamos usando o IMC como um valor \emph{proxy} para variáveis relativas
ao estilo de vida dos alunos (apesar das limitações conhecidas deste
indicador
{[}\href{http://www.nytimes.com/interactive/projects/cp/summer-of-science-2015/latest/how-often-is-bmi-misleading}{2},\href{http://science.sciencemag.org/content/341/6148/856.summary}{3},\href{http://www.medicalnewstoday.com/articles/265215.php}{4}{]}.
Neste caso, é razoável supor que a divisão da análise em duas
sub-populações (masculina e feminina) seja interessante.

Os dados relativos ao semestre 2016-2 estão tabulados no arquivo
\href{https://raw.githubusercontent.com/fcampelo/Design-and-Analysis-of-Experiments/master/data\%20files/imc_20162.csv}{imc\_20162.csv},
disponível na pasta \emph{data files} do repositório Github da
disciplina; e os dados da turma 2017-2 estão disponíveis na mesma pasta,
no arquivo
\href{https://raw.githubusercontent.com/fcampelo/Design-and-Analysis-of-Experiments/master/data\%20files/CS01_20172.csv}{CS01\_20172.csv}.

Note que o arquivo relativo a 2016-2 contém também dados de uma turma de
graduação, e que os dois arquivos (2016-2 e 2017-2) estão em formatos
ligeiramente diferentes. É parte da tarefa deste estudo de caso isolar
os dados de interesse e consolidar os mesmos de forma a realizar a
análise de forma correta.

\subsection{Atividades}\label{atividades}

\begin{itemize}
\tightlist
\item
  Definição das hipóteses de teste (qual a hipótese nula? Qual a
  alternativa? Que tipo de teste utilizar?);
\item
  Teste de hipóteses;
\item
  Estimação do tamanho do efeito e do intervalo de confiança na grandeza
  de interesse;
\item
  Verificação e discussão das premissas do teste;
\item
  Derivação de conclusões e recomendações.
\item
  Discussão sobre a potência do teste (se aplicável).
\item
  Discussão sobre possíveis formas de melhorar este experimento.
\end{itemize}

Atenção: é interessante neste caso separar as amostras por sexo
(masculino/feminino) e realizar testes independentes para as populações
correspondentes.

\subsection{Relatório}\label{relatorio}

Cada grupo deverá entregar um relatório detalhando o experimento e a
análise dos dados. O relatório será avaliado de acordo com os seguintes
critérios:

\begin{itemize}
\tightlist
\item
  Obediência ao formato determinado (ver abaixo);
\item
  Reproducibilidade dos resultados;
\item
  Qualidade técnica;
\item
  Estrutura da argumentação;
\item
  Correto uso da linguagem (gramática, ortografia, etc.);
\end{itemize}

O relatório deve \emph{obrigatoriamente} ser produzido utilizando
\href{http://rmarkdown.rstudio.com}{R Markdown} (opcionalmente
utilizando estilos distintos, como o do presente documento), e deve
conter todo o código necessário para a reprodução da análise obtida,
embutido na forma de blocos de código no documento. Os grupos devem
enviar:

\begin{itemize}
\tightlist
\item
  O arquivo \textbf{.Rmd} do relatório.
\item
  O arquivo de dados utilizado.
\end{itemize}

O arquivo \textbf{.Rmd} deve ser capaz de ser compilado em um pdf sem
erros, e deve assumir que o arquivo de dados se encontra no mesmo
diretório do arquivo do relatório. Modelos de estudos de caso estão
disponíveis
\href{https://github.com/fcampelo/Design-and-Analysis-of-Experiments/tree/master/Grading/Report\%20template}{aqui}
e
\href{https://github.com/fcampelo/Design-and-Analysis-of-Experiments/tree/master/Grading/Case\%20Studies/Report\%20Examples}{aqui}.
Caso a equipe deseje utilizar o estilo do presente documento, pode
consultar seu código-fonte
\href{https://raw.githubusercontent.com/fcampelo/Design-and-Analysis-of-Experiments/master/Semester\%20files/2017-2\%20(PT)/Case\%20Studies/CS02.Rmd}{aqui}
(note que o mesmo requer a instalação do pacote \emph{pinp}).

\textbf{Importante}: Salve seu arquivo \textbf{.Rmd} em UTF-8 (para
evitar erros na compilação em outros sistemas). \textbf{Importante}:
Inclua no relatório os papéis desempenhados por cada membro da equipe
(Relator, Verificador etc.)

Relatórios serão aceitos em português, inglês ou espanhol.

\subsection{Entrega}\label{entrega}

Os arquivos deverão ser enviados via \emph{e-mail} para o endereço
\url{fcampelo@ufmg.br}. O título do e-mail deve seguir o padrão
``\textbf{{[}EEE933\_2017-2\_EC02{]} Nome\_da\_equipe}'' (sem as aspas).
A data-limite para o recebimento dos arquivos é \textbf{quarta-feira
(17/10) às 23:55h}

%\showmatmethods

\pnasbreak 




\end{document}

